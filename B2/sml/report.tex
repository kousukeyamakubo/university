\documentclass[dvipdfmx]{jarticle}
\usepackage{graphicx}
\usepackage{amsmath}
\usepackage[top=30truemm,bottom=30truemm,left=25truemm,right=25truemm]{geometry}
\usepackage{listings,jvlisting}
\usepackage{amssymb}

\lstset{
  basicstyle={\ttfamily},
  identifierstyle={\small},
  commentstyle={\smallitshape},
  keywordstyle={\small\bfseries},
  ndkeywordstyle={\small},
  stringstyle={\small\ttfamily},
  frame={tb},
  breaklines=true,
  columns=[l]{fullflexible},
  numbers=left,
  xrightmargin=0zw,
  xleftmargin=3zw,
  numberstyle={\scriptsize},
  stepnumber=1,
  numbersep=1zw,
  lineskip=-0.5ex
}

\begin{document}

\begin{titlepage}
    \begin{center}
        \vspace*{60pt}
        {\LARGE プログラミングD SML レポート}
        \vspace*{240pt}\\
        \begin{tabular}{rl}
            担当教員 & 小南大智\\
            提出日 & \today\\
            氏名 & 山久保孝亮\\
            学籍番号 & 09B22084\\
            メールアドレス & u327468b@ecs.osaka-u.ac.jp
        \end{tabular}
    \end{center}
\end{titlepage}

\section{課題3のプログラムの説明}
課題3ではCOMPの計算の処理を足し算のほかに引き算,掛け算,割り算を追加した.これを実現するために以下の二つの変更を加えた.
\begin{itemize}
    \item 関数EXPから関数COMPに移動する条件の変更.
    \item それぞれの演算の処理の実装.
\end{itemize}
以下でその詳細について述べる.
\subsection{条件の変更}
関数EXPから関数COMPに移動する条件を以下のように変更した.
\begin{lstlisting}[caption=COMPに移動する条件,label=fuga]
    else if h = "+" orelse h = "-" orelse h = "*" orelse h = "/" then COMP (h::t)
\end{lstlisting}
課題2では足し算の演算を実行するときの処理を参考に,hがそれぞれの演算子であったときに関数COMPに移動するようにした.
\subsection{処理の実装}
各演算の処理の実装は以下のようになる.
\begin{lstlisting}[caption=処理の実装,label=fuga]
    if h = "+" then
                let
                    val (v1,t1) = EXP t
                    val (v2,t2) = EXP t1
                in
                    (v1 + v2, t2)
                end
            else if h = "-" then
                let
                    val (v1,t1) = EXP t
                    val (v2,t2) = EXP t1
                in
                    (v1 - v2, t2)
                end 
            else if h = "*" then
                let
                    val (v1,t1) = EXP t
                    val (v2,t2) = EXP t1
                in
                    (v1 * v2, t2)
                end 
            else if h = "/" then
                let
                    val (v1,t1) = EXP t
                    val (v2,t2) = EXP t1
                in
                    (v1 div v2, t2)
                end 
            else raise SyntaxError
\end{lstlisting}
足し算の演算の処理と基本的な構造はすべて同じである.8,15,22行目でhがそれぞれどの演算子を表すのかを判定し,hに合った処理を実行する.そして計算結果と残りの文字列の組を返す.
ただし,smlにおける整数同士の割り算は"/"ではなく"div"を使うので27行目のような値の組を返す.
\section{課題4のプログラムの説明}

\section{課題5のプログラムの説明}

\section{課題6のプログラムの説明}

\section{拡張機能の説明}

\end{document}