\documentclass[dvipdfmx]{jarticle}
\usepackage{graphicx}
\usepackage[top=30truemm,bottom=30truemm,left=25truemm,right=25truemm]{geometry}
\usepackage{listings,jvlisting}

\lstset{
  basicstyle={\ttfamily},
  identifierstyle={\small},
  commentstyle={\smallitshape},
  keywordstyle={\small\bfseries},
  ndkeywordstyle={\small},
  stringstyle={\small\ttfamily},
  frame={tb},
  breaklines=true,
  columns=[l]{fullflexible},
  numbers=left,
  xrightmargin=0zw,
  xleftmargin=3zw,
  numberstyle={\scriptsize},
  stepnumber=1,
  numbersep=1zw,
  lineskip=-0.5ex
}

\begin{document}
\begin{titlepage}
    \begin{center}
        {\huge 情報科学実験A レポート4}
        \vspace{180pt}\\
        \begin{tabular}{rl}
            氏名 & 山久保孝亮\\
            所属 & 大阪大学基礎工学部情報科学科ソフトウェア科学コース\\
            学籍番号 & 09B22084\\
            提出日 & \today\\
            担当教員 & 繁田 浩功/桝井 晃基
        \end{tabular}
    \end{center}
\end{titlepage}
\section{実験ペアの名前および学籍番号}
\begin{table}[h]
    \centering
    \begin{tabular}{|c|c|}
        \hline
        学籍番号 & 名前\\\hline\hline
        09B22083 & 安川雄輝\\\hline
    \end{tabular}
    \caption{ペアの学籍番号と名前}
\end{table}
\section{送受信回路のステート・マシンの概要図}
\section{送受信回路の設計内容}
\section{テストベンチによる動作確認内容・結果}
\section{非同期通信回路作成における注意点などの考察}
\section{教員・TAによる動作確認時刻}
1月日時分
\end{document}