\documentclass[dvipdfmx]{jarticle}
\usepackage{graphicx}
\usepackage[top=30truemm,bottom=30truemm,left=25truemm,right=25truemm]{geometry}
\usepackage{listings,jvlisting}
\usepackage{url}

\lstset{
  basicstyle={\ttfamily},
  identifierstyle={\small},
  commentstyle={\smallitshape},
  keywordstyle={\small\bfseries},
  ndkeywordstyle={\small},
  stringstyle={\small\ttfamily},
  frame={tb},
  breaklines=true,
  columns=[l]{fullflexible},
  numbers=left,
  xrightmargin=0zw,
  xleftmargin=3zw,
  numberstyle={\scriptsize},
  stepnumber=1,
  numbersep=1zw,
  lineskip=-0.5ex
}

\makeatletter
\newcommand{\subsubsubsection}{\@startsection{paragraph}{4}{\z@}%
  {1.0\Cvs \@plus.5\Cdp \@minus.2\Cdp}%
  {.1\Cvs \@plus.3\Cdp}%
  {\reset@font\sffamily\normalsize}
}
\makeatother
\setcounter{secnumdepth}{4}

\begin{document}
\begin{titlepage}
    \begin{center}
        {\huge 情報科学演習D 課題2レポ―ト}
        \vspace{180pt}\\
        \begin{tabular}{rl}
            氏名 & 山久保孝亮\\
            所属 & 大阪大学基礎工学部情報科学科ソフトウェア科学コース\\
            メールアドレス & u327468b@ecs.osaka-u.ac.jp\\
            学籍番号 & 09B22084\\
            提出日 & \today\\
            担当教員 & 桝井晃基 松本真佑
        \end{tabular}
    \end{center}
\end{titlepage}
\section{システムの仕様}
課題2の外部使用は以下のとおりである.
\begin{itemize}
  \item 第一引数で指定されたtsファイルを読み込む.
  \item 構文が正しい場合は文字列"OK"を返し,正しくない場合は"Syntax error: line"という文字列を返す.ただし,lineの部分には最初に誤りがあった行番号を表し,複数の誤りがあったとしても最初の誤りのみ出力する.また,入力ファイルが見つからない場合は"File not found"を返す.
\end{itemize}
\section{課題達成の方針と設計}
今回の課題の方針としては,指導書に記述されている構文定義の左辺に対応するメソッドをそれぞれ作成し,再帰降下法により構文解析をした.
それぞれのメソッドは全てint型で定義し,返り値が0のときそのメソッド内の構文定義は正しいことを表し,返り値が0以外のとき構文の誤りがあった行番号を表す.具体的には,次のトークンとそのメソッド内で期待されるトークンを比較し一致していれば構文が正しく,一致していなければその時点での行番号を返す.\\
また,文の構文定義など,次の一つのトークンを読んだだけではif-elseの文なのかifだけの文なのか判断できないような場合は左くくり出しを行う.具体的には文の構文定義では,ifから複合文までの定義が同じであるため,そこまでは同じ条件分岐として扱い,そこから再び次のトークンを読むことによる構文の判定を行う.
また,行番号はどのメソッドでも使用できるようにするためグローバル変数としてts\_line\_numberを定義している.また,この変数は0に初期化されている.
\section{実装プログラム}
run()メソッドではtsファイルの内容を列ごとにリストbufferに格納し,今回の構文解析のプログラムが記述されたparse()メソッドに引数として渡す.このparse()の返り値が0であればこのメソッドが正常に終了したことを表すので文字列"OK"を,返り値が0でなければその値が
エラーが最初に発生した行なのでこれを外部仕様に記述した文字列とともに出力する.\\
parse()では,リストbufferのts\_line\_number番目の要素を文字列lineに格納する.このとき,ts\_line\_numberの値を1インクリメントする.課題1より,この文字列はトークン列とその情報を表すが,すべてタブ文字で区切られているため
split()を使って配列partsに格納する.今後,次のトークンを取り出すための処理は以上の操作と同じ内容を繰り返す.
その後program()を呼び出すが,この処理は以下の3.2の処理と同じであるため後述する.そして,ts\_line\_numberとbufferの長さが同じになれば0を返す.
\\2の方針に従ってプログラムを実装すると,今回のプログラムは以下のパターンのプログラムに分類できる.
\begin{enumerate}
  \item 次に期待されているトークンが""で囲まれた予約語である場合.
  \item 次に期待されているトークンが構文要素名である場合.
  \item 中括弧\{...\}で囲まれた0回以上の繰り返しの場合.
  \item 角括弧[...]で囲まれた0回または1回の出現の場合.
  \item 次のトークンを読んだうえでどの句を選択するかを決定する必要がある場合.
\end{enumerate}
以下でそれぞれの実装方法について記述する.
\subsection{次のトークンが予約語である場合}
このパターンのプログラムを一般化したものは以下のようになる.
\begin{lstlisting}
String line = buffer.get(ts_line_number++);
String[] parts = line.split("\t");

if (!parts[2].equals("43")) {
    return Integer.parseInt(parts[3]); 
}
\end{lstlisting}
上述のとおり,tsファイルから取り出されたトークンとその情報(pasファイル上でのトークン,字句解析器上でのトークン名,トークンID,行番号)はそれぞれpartsの各要素に格納されている.
今回の課題を通して,これらの内用いられるのはトークンIDと行番号の二つ,即ちparts[2]とparts[3]である.parts[2]即ちトークンIDが想定されたトークンIDと一致しているかどうかをif文を用いて判定している.
一致していなければこの時点でこのときの行番号parts[3]がreturnされる.そして一致していれば何もreturnされずに次の処理に進む.
\subsection{次のトークンが構文要素名である場合}
このパターンのプログラムを一般化したものは以下のようになる.
\begin{lstlisting}
int result = parseBlock(buffer);
if (result != 0) {
    return result;
}
\end{lstlisting}
ここでsyntax\_definition()は,それぞれの構文定義のメソッドを一般化したものである.
int型の変数resultにint型で定義されたそれぞれのメソッドの返り値が格納されるので,resultが0でない即ち呼び出したメソッド内で構文定義に誤りがあった場合にはその値が出力する行番号なのでreturnする.
0であれば何もreturnされずに次の処理に進む.
\subsection{0回以上の繰り返しの場合}
このパターンのプログラムを一般化したものは以下のようになる.
\begin{lstlisting}
while(true) {
  line = buffer.get(ts_line_number++);
  parts = line.split("\t");
  if(!parts[2].equals("41")) {
    ts_line_number--;
    break;
  }
//
以下にさらに処理が続く
}
\end{lstlisting}
\subsection{0回または1回の出現の場合}
このパターンのプログラムを一般化したものは以下のようになる.
\begin{lstlisting}
  
\end{lstlisting}
\subsection{次のトークンを読んだうえでどの句を選択するかを決定する必要がある場合}
\section{考察}
\section{感想}

\begin{thebibliography}{1}
    \bibitem{1} 2024年度情報科学演習D指導書
\end{thebibliography}
\end{document}