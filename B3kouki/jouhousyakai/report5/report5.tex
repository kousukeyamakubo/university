\documentclass[dvipdfmx]{jarticle}
\usepackage{graphicx}
\usepackage[top=30truemm,bottom=30truemm,left=25truemm,right=25truemm]{geometry}
\usepackage{listings,jvlisting}
\usepackage{url}

\lstset{
  basicstyle={\ttfamily},
  identifierstyle={\small},
  commentstyle={\smallitshape},
  keywordstyle={\small\bfseries},
  ndkeywordstyle={\small},
  stringstyle={\small\ttfamily},
  frame={tb},
  breaklines=true,
  columns=[l]{fullflexible},
  numbers=left,
  xrightmargin=0zw,
  xleftmargin=3zw,
  numberstyle={\scriptsize},
  stepnumber=1,
  numbersep=1zw,
  lineskip=-0.5ex
}

\begin{document}
\begin{titlepage}
    \begin{center}
        {\huge 情報技術者と社会 第5回レポート}
        \vspace{180pt}\\
        \begin{tabular}{rl}
            氏名 & 山久保孝亮\\
            所属 & 大阪大学基礎工学部情報科学科ソフトウェア科学コース\\
            メールアドレス & u327468b@ecs.osaka-u.ac.jp\\
            学籍番号 & 09B22084\\
            提出日 & \today\\
        \end{tabular}
    \end{center}
\end{titlepage}
\section{著作権保護期間延長の経緯}
著作権保護期間が2018年に50年から70年に延長された経緯としては以下の通りである.\cite{1}
\begin{enumerate}
    \item 1990年代に欧州で著作権保護期間が70年に延長される.
    \item 2006年に権利者団体の要望により文化庁が期間延長を検討する.しかし,権利処理の複雑化や遺族の収入が増加しないこと,民間の負担増などの理由により2010年に見送られる.
    \item TPPの米国提案の知的条項案にTPPの加盟国は保護期間を死後70年に延長するという要求が入っていた.これにより国内での議論が再び高まり,TPP特別法を批准するための特別法を前倒しで整備した.
    \item TPP関連法が成立した1か月後に米国でトランプ大統領が当選し,TPPから米国が離脱したため日本政府が主導して米国を復帰させるために保護期間延長の部分が凍結される.
    \item 政府がTPPと並行して交渉していたEUとのEPA協議中に,保護期間延長で合意していたことが公表された.そして,2018年6月にEPAが発行されていない状態で保護期間延長を含むTPP整備法を可決し,12月から発行と同時に施行された.
\end{enumerate}
\section{国家間で保護期間を統一するメリットとデメリット}
国家間で保護期間を統一するメリットとしては,インターネット等で保護期間が切れている国にサーバを置いて著作物を発信すれば保護期間が切れていない国からでもダウンロードできてしまうという点があげられる.\cite{2}
\\また,統一することによるデメリットとしては,保護期間が延長された国の人にとって,創作者がアクセスできる情報量が減少し社会全体の文化の発展が阻害される可能性があるという点が挙げられる.
\section{著作権保護期間延長について}
著作権保護期間延長の利点は以下の通りである.\cite{3}
\begin{itemize}
    \item 創作者の意図や尊厳が保たれる期間が延びる.
    \item 将来的に創作者が得る収入量が増加する可能性がある.
\end{itemize}
また,著作権保護期間延長の欠点は以下の通りである.
\begin{itemize}
    \item 創作者がアクセスできる情報量が減少し社会全体の文化の発展が阻害される可能性がある
    \item 著作権は相続人全員の共有が原則であり,全員の同意がなければ利用できないため,保護期間が長期化すると権利者の確認等の権利処理の負担が増加してしまう.
\end{itemize}
\section{著作権保護期間短縮について}
著作権保護期間短縮の利点は以下の通りである.\cite{4}
\begin{itemize}
    \item 社会が享受できる創作の量が増加する.
\end{itemize}
また,著作権保護期間短縮の欠点は以下の通りである.
\begin{itemize}
    \item 創作者は経済的利益を得られる期間が減少してしまう.
\end{itemize}
\section{私の考え}
以上のことを通して,私は適正な保護期間については各著作権が発生させている利益に応じて決定すべきであると考える.
上記の通り,著作権保護期間の延長の是非については経済的な理由と文化的な理由が対立していることから発生している.
経済的な立場から言えばミッキーマウスなどの著作物とほかの著作物では生み出す利益が異なるためそれを他と同じように扱ってしまうと新たな創造意欲を失わせかねない.
文化的な立場から言えば,いかなる作品も新たな創造性を掻き立てる可能性はあるため,なるべく人々が情報として得やすい状態にしておくべきである.
したがって,有名でない作品即ち経済的に利益を生み出していない作品へのアクセスの容易さを担保するためにそういった作品の著作権がなるべく早く切れるようにし,秀でた作品などは注目を集めるので
文化的にも経済的にもメリットを共有できるため利益によって期間を決定することは理に適っていると私は考えた.しかし,この方法の問題点としては時代を先どっている作品のように,何十年経過してから評価されるような作品の著作者が利益を得ることができず,
結果的に作品の種類の幅が狭まってしまいかねないという点があげられる.
\begin{thebibliography}{99}
    \bibitem{1} \url{https://www.jstage.jst.go.jp/article/jsda/3/3/3_327/_pdf/-char/ja} 11/12アクセス
    \bibitem{2} \url{https://www.mext.go.jp/b_menu/shingi/bunka/gijiroku/021/07092813/002/004.htm#:~:text=%E5%BB%B6%E9%95%B7%E3%81%99%E3%82%8B%E3%81%93%E3%81%A8%E3%81%AB%E3%82%88%E3%81%A3%E3%81%A6%E8%A8%B1%E8%AB%BE,%E3%82%92%E5%A5%AA%E3%81%86%E3%81%93%E3%81%A8%E3%81%AB%E3%81%AA%E3%82%8B%E3%80%82} 11/13アクセス
    \bibitem{3} \url{https://www.bunka.go.jp/seisaku/bunkashingikai/chosakuken/hosei/h21_03/pdf/shiryo_3.pdf} 11/13アクセス
    \bibitem{4} \url{https://p2ptk.org/copyright/1590} 11/13アクセス
\end{thebibliography}
\end{document}