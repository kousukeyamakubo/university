\documentclass[dvipdfmx]{jarticle}
\usepackage{url}
\title{情報技術者と社会第4回レポート}
\author{ソフトウェア科学コース\\09B22084山久保孝亮}
\date{\today}

\begin{document}
\maketitle
\section{政府などの公的機関による監視に対しての考え}
私は政府などの公的機関による監視は犯罪の操作だけでなく感染症等が流行した際にも使用が許容されるべきであると考えた.
理由としては,監視システムによって人の移動や密度等を特定することができ,また医療資源などのリソース配分を判断する際に全体の様子を眺めることができるようになる.
具体的には新型コロナウイルスが流行した際にはHER-SYSというシステムによって感染者の情報等を管理していた\cite{1}が,犯罪捜査等で用いられている監視カメラの情報等を活用することも許容すべきであると考える.
これにより感染者とみられる人を監視カメラで追って正確な移動経路を把握できるようになる.
\section{企業の情報収集とその利用に対しての考え}
企業の情報収集に関しては漏洩したとしても購入者個人に影響が及ばない範囲の情報で利用されるべきであると考える.その理由としては,企業が利益を追求するために個人情報を利用することによってコストを削減しながら効率的にユーザの需要を把握し,また利用者側もサービス向上による恩恵を受けられるためである.
今回私が考える具体的な影響が及ばない範囲の情報は,その会社の商品に関する顧客の意見等である.逆に,住所などの情報を含める際は漏洩しても問題ないようにデータを加工するなどするべきであると考える.
\begin{thebibliography}{99}
    \bibitem{1} https://www.mhlw.go.jp/content/10900000/000728154.pdf 10/28アクセス

\end{thebibliography}
\end{document}