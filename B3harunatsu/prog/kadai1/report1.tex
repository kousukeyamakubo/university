\documentclass[dvipdfmx]{jarticle}
\usepackage{url}
\usepackage[top=30truemm,bottom=30truemm,left=25truemm,right=25truemm]{geometry}
\title{プログラム設計レポート1}
\author{ソフトウェア科学コース\\09B22084山久保孝亮}
\date{\today}

\begin{document}
\maketitle
\section{課題1.1}
私が関わったソフトウェア開発で最も大きな規模のプログラムはPBLの授業で作成したマッチングプログラムである.
プログラムサイズは約800行程度で,開発言語はPythonである.開発環境はVSCodeで,人数は6人,期間は半年である.設計は
まず最初に必要な機能を挙げ,それぞれの機能を関数として実装するようにした.これによって各々が作業に取り組むことができた.
\section{課題1.2}
\subsection{GOTO文}
GOTO文を多用することによって,読みにくいコードになってしまう.具体的には通常のプログラムだと上から下へ処理が順番に流れていくが
GOTO文は突然別の処理に飛んでしまったりしてしまうため,処理の流れがつかみにくくなってしまう.ただし,多重ループの一斉脱出などGOTO文が有効になる場合もあるので
一概に禁止にすればよいという訳ではない.
\subsection{手続きの長さ}
手続きが長くなりすぎると,すべてのコードに目を通すのに時間がかかるので可読性が低くなったり,どの箇所を改善すればよいかを特定しにくくなるので保守性が低くなったりしてしまう.
したがって一つの手続きの長さをあらかじめ特定の長さに制限しておく必要がある.
\subsection{コメント}
可読性を向上させるためになるべくたくさん記載すべきであるが,その書き方は一定の決まりを設けるなどの工夫が必要である.
\subsection{変数名}
変数が使用される際の役割に適した変数名を付ける必要がある.より具体的な名前を付けた方がより分かりやすくなる.
長い名前の入力は大変なので,既出の変数の予測変換など,開発エディタの機能等を活用してタイプミス等を減らしていく必要がある.
\section{課題1.6}
\subsection{PASCALコンパイラ}
メモリ消費量などの動作視点と,階層化設計などの構造視点の設計図が重要である.
\subsection{プリンタ制御用プログラム}
入力でどのようにモータ等を操作するかについての機能視点,どのような順序で動作を実行していくか,すなわちどのような状態遷移をしていくのかについての動作視点,
機能が達成されるまでの時間についての時間視点などが重要になる.
\subsection{数値計算ライブラリ}
入出力に関する機能視点とアルゴリズムなどの内部処理に関する動作視点が重要である.

\end{document}